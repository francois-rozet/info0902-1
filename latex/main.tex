\documentclass[a4paper, 12pt]{article}

\def\languages{english, french}

%%%%%%%%%%%%%%%%%%% Libraries

\input{./include/libraries/default.tex}
\input{./include/libraries/informatics.tex}
\input{./include/libraries/mathematics.tex}
\input{./include/libraries/units.tex}

%%%%%%%%%%%%%%%%%%% Titlepage

\def\logopath{./resources/pdf/logo.pdf}
\def\toptitle{Université de Liège}
\title{Recherche de motifs dans \\[0.25em] des séquences d'ADN}
\def\subtitle{INFO0902-1 -- Structure de données et algorithmes}
%\def\authorhead{}
\author{François \textsc{Rozet} (s161024)\\}
%\def\rightauthorhead{}
%\def\rightauthor{}
\def\context{Bachelier Ingénieur civil}
\date{Année académique 2018-2019}

%%%%%%%%%%%%%%%%%%%

\usepackage[ruled]{algorithm}
\usepackage[noend]{algpseudocode}

\floatname{algorithm}{Pseudo-code}

\algrenewcommand\alglinenumber[1]{\footnotesize #1}

\newcommand{\algorithmicerror}{\textbf{error}}
\newcommand{\Error}{\algorithmicerror{}}

\newcommand{\algorithmicbreak}{\textbf{break}}
\newcommand{\Break}{\algorithmicbreak{}}

\renewcommand{\algorithmiccomment}[1]{// #1}

\newcommand{\To}{\textbf{ to }}
\renewcommand{\Call}[2]{\textsc{#1}(#2)}

%%%%%%%%%%%%%%%%%%%

\begin{document}
	\input{./include/titlepages/default.tex}
	\romantableofcontents
	\section{Analyse théorique}
	\subsection{Recherche dichotomique par force brute ou table de hachage}
	\begin{algorithm}[H]
		\begin{algorithmic}[1]
			\Function{Dichotomic-Search}{$X, Y$}
				\State $t = NIL$
				\State $p = 1; \, r = \min(X.length, Y.length)$
				\While{$p \leq r$}
					\State $q = \left\lfloor \frac{p + r}{2} \right\rfloor$
					\State $s = $ \Call{Brute-Force-Aux}{$X, Y, q$}
					\If{$s \neq NIL$}
						\State $p = q + 1$
					\Else
						\State $r = q - 1$
						\State $t = s$
					\EndIf
				\EndWhile
				\State \Return $t$
			\EndFunction
		\end{algorithmic}
		\caption{Recherche dichotomique par force brute}
		\label{pc:dichotomic_search}
	\end{algorithm}
	\vspace{-1em}
	\begin{algorithm}[H]
		\begin{algorithmic}[1]
			\Function{Hash-Table-Search}{$X, Y$}
				\State $t = NIL$
				\State $p = 1; \, r = \min(X.length, Y.length)$
				\While{$p \leq r$}
					\State $q = \left\lfloor \frac{p + r}{2} \right\rfloor$
					\State $s = $ \Call{Hash-Table-Aux}{$X, Y, q$}
					\If{$s \neq NIL$}
						\State $p = q + 1$
					\Else
						\State $r = q - 1$
						\State $t = s$
					\EndIf
				\EndWhile
				\State \Return $t$
			\EndFunction
		\end{algorithmic}
		\caption{Recherche dichotomique par table de hachage}
		\label{pc:hash_table_search}
	\end{algorithm}
	\vspace{-1em}
	\begin{algorithm}[H]
		\begin{algorithmic}[1]
			\Function{Brute-Force-Aux}{$X, Y, l$}
				\For{$i = 1 \To  X.length - l + 1$}
					\For{$j = 1 \To Y.length - l + 1$}
						\For{$k = 1 \To l$}
							\If{$X[i + k - 1] \neq Y[j + k - 1]$}
								\State $k = 0$
								\State \Break
							\EndIf
						\EndFor
						\If{$k \neq 0$}
							\State \Return $(l, i, j)$
						\EndIf
					\EndFor
				\EndFor
				\State \Return $NIL$
			\EndFunction
		\end{algorithmic}
		\caption{Recherche auxiliaire par force brute}
		\label{pc:brute_force_aux}
	\end{algorithm}
	\vspace{-1em}
	\begin{algorithm}[H]
		\begin{algorithmic}[1]
			\Function{Hash-Table-Aux1}{$X, Y, l$}
				\State Let $H$ be a new hash table
				\For{$i = 1 \To X.length - l + 1$}
					\State $f = $ \Call{Encode}{$X, i, l$}
					\State \Call{Hash-Insert}{$H, f, i$}
				\EndFor
				\For{$j = 1 \To Y.length - l + 1$}
					\State $f = $ \Call{Encode}{$Y, j, l$}
					\State $i = $ \Call{Hash-Search}{$H, f$}
					\If{$i \neq NIL$}
						\State \Return $(l, i, j)$
					\EndIf
				\EndFor
				\State \Return $NIL$
			\EndFunction
		\end{algorithmic}
		\caption{Recherche auxiliaire par table de hachage sans calcul incrémental}
		\label{pc:hash_table_aux1}
	\end{algorithm}
	\vspace{-1em}
	\begin{algorithm}[H]
		\begin{algorithmic}[1]
			\Function{Hash-Table-Aux2}{$X, Y, l$}
				\State Let $H$ be a new hash table
				\State $b = 4; \, p = b^{l-1}$
				\State $f = $ \Call{Encode}{$X, 1, l$}
				\State \Call{Hash-Insert}{$H, f, 1$}
				\For{$i = 2 \To X.length - l + 1$}
					\State $f = (f - X[i - 1] \cdot p) \cdot b + X[i + l - 1]$
					\State \Call{Hash-Insert}{$H, f, i$}
				\EndFor
				\State $f = $ \Call{Encode}{$Y, 1, l$}
				\State $i = $ \Call{Hash-Search}{$H, f$}
				\If{$i \neq NIL$}
					\State \Return $(l, i, 1)$
				\EndIf
				\For{$j = 2 \To Y.length - l + 1$}
					\State $f = (f - Y[j - 1] \cdot p) \cdot b + Y[j + l - 1]$
					\State $i = $ \Call{Hash-Search}{$H, f$}
					\If{$i \neq NIL$}
						\State \Return $(l, i, j)$
					\EndIf
				\EndFor
				\State \Return $NIL$
			\EndFunction
		\end{algorithmic}
		\caption{Recherche auxiliaire par table de hachage avec calcul incrémental}
		\label{pc:hash_table_aux2}
	\end{algorithm}
	\vspace{-1em}
	\begin{algorithm}[H]
		\begin{algorithmic}[1]
			\Function{Encode}{$X, i, l$}
				\State $b = 4$
				\State $f = 0$
				\For{$j = 1 \To l$}
					\State $f = f \cdot b + X[i + j - 1]$
				\EndFor
				\State \Return $f$
			\EndFunction
		\end{algorithmic}
		\caption{Fonction d'encodage}
		\label{pc:encode}
	\end{algorithm}
	\subsection{Fonction de coût}
	\begin{equation}
		C[i,j ] =
		\left\{
		\begin{aligned}
			& C[i - 1, j - 1] + 1 && \text{si } i, j > 0 \text{ et } X[i] = Y[j] \\
			& 0  && \text{sinon} \\
		\end{aligned}
		\right.
	\end{equation}
	\subsection{Programmation dynamique}
	\begin{algorithm}[H]
		\begin{algorithmic}[1]
			\Function{Dynamic-Search}{$X, Y$}
				\State Let $C[0 \ldots X.length,0 \ldots Y.length]$ be a new table
				\State $l = 0$
				\For{$i = 0 \To X.length$}
					\For{$j = 0 \To Y.length$}
						\If{$i > 0$ and $j > 0$ and $X[i] == Y[j]$}
							\State $C[i,j] = C[i - 1,j - 1] + 1$
							\If{$C[i, j] > l$}
								\State $l = C[i, j]$
								\State $i_{end} = i$
								\State $j_{end} = j$
							\EndIf
						\Else{} $C[i,j] = 0$
						\EndIf
					\EndFor
				\EndFor
				\If{$l \neq 0$}
					\State \Return $(l, i_{end} - l + 1, j_{end} - l + 1)$
				\EndIf
				\State \Return $NIL$
			\EndFunction
		\end{algorithmic}
		\caption{Recherche par programmation dynamique}
		\label{pc:dynamic_search}
	\end{algorithm}
	\subsection{Complexité}
	Soit $n$ la longueur des deux séquences et $\alpha n$ la longueur de leur sous-séquence contiguë commune la plus longue.
	\begin{enumerate}[label=(\alph*)]
		\item La solution par force brute est composée de quatre boucles imbriquées. Les boucles initiale et intermédiaires exécutent toujours de l'ordre de $(1 - \alpha) n$ itérations. La boucle finale, quant à elle, en exécute de l'ordre de $\alpha n$. Les opérations à l'intérieur de ces boucles étant toutes $\mathcal{O}(1)$, la complexité en temps est
		\begin{equation}
			\mathcal{O}\rbk{\alpha (1 - \alpha)^3 n^4} = \mathcal{O}\rbk{n^4}
		\end{equation}
		Néanmoins, lorsque $\alpha n$ tend vers $n$ ($\alpha \to 1$), les trois premières boucles voient leur nombre d'itérations tendre vers $1$. Ainsi, la complexité en temps est $\Omega(n)$. On trouve aussi que $\alpha = \frac{1}{4}$ est le pire cas, bien qu'il soit équivalent au cas moyen en terme de complexité. \par
		Concernant la complexité en espace, selon ce qu'il a été décidé de renvoyer, elle peut être $\Theta(1)$ ou $\Theta(\alpha n)$. Si la recherche renvoie un triplet $(l,i,j)$ ($l$ la longueur, $i$ l'indice de départ dans $G1$ et $j$ l'indice de départ dans $G2$), la complexité sera $\Theta(1)$. À l'inverse, si la recherche renvoie la sous-séquence sous forme d'une table\footnote{Les pseudo-codes ont été rédigés selon cette politique.}, la complexité sera $\Theta(\alpha n)$.
		\item Comme la solution par force brute, la recherche dichotomique est composée de quatre boucles imbriquées. Cependant, ici, la boucle initiale s'arrête lorsque un interval de taille $n$ a été divisé \emph{suffisamment} de fois, \cad{} $\log_2(n)$ fois. De plus, les boucles intermédiaires sont maintenant en $\mathcal{O}(n)$. Ainsi, la complexité en temps est
		\begin{equation}
			\mathcal{O}\rbk{\alpha n^3 \log(n)} = \mathcal{O}\rbk{n^3 \log(n)}
		\end{equation}
		Pour cette solution, lorsque $\alpha n$ tend vers $n$, les boucles intermédiaires ne voient pas leur nombre d'itérations tendre vers $1$. Cependant, lorsque $\alpha n = 0$, la boucle la plus profonde n'effectue plus qu'une seul itération. Ainsi, la complexité en temps est $\Omega(n^2 \log(n))$. Le pire cas est équivalent au cas moyen. \par
		Concernant la complexité en espace, cette solution est équivalent à la force brute.
		\item La solution par table de hachage ne possède que deux boucles imbriquées qui correspondent aux deux premières de la recherche dichotomique. Néanmoins, sans calcul incrémental, l'opération d'encodage exécute $\mathcal{O}(n)$ opérations à chaque appel. On suppose aussi que l'insertion et la lecture dans la table de hachage sont faites en $\mathcal{O}(1)$. Dès lors, la complexité en temps est
		\begin{equation}
			\Theta\rbk{n^2 \log(n)}
		\end{equation}
		Concernant la complexité en espace, à chaque appel de la fonction auxiliaire pour une taille $l$ de sous-séquence, il est nécessaire d'initialiser une table de hachage sur un univers $U = \{0,1,\ldots,4^l-1\}$ afin d'y stocker $n - l + 1$ sous-séquences. Pour conserver un accès (insertion, recherche et suppression) aux éléments de la table en $\mathcal{O}(1)$, il est nécessaire d'avoir $m = \Omega(n - l)$ où $m$ est sa taille\footnote{Dans le cas d'une table de hachage par adressage direct, la condition $m \geq n - l + 1$ doit aussi être respectée.}. Or, la valeur de départ de $l$ est $\frac{n}{2}$, dès lors, la complexité en espace de la solution par table de hachage est $\Theta(n)$.
		\item Avec le calcul incrémental, l'encodage ne demande plus que $\mathcal{O}(1)$ opérations par appel. Ainsi, la complexité en temps devient
		\begin{equation}
			\Theta\rbk{n \log(n)}
		\end{equation}
		La complexité en espace n'a, quant à elle, pas changé.
		\item La solution par programmation dynamique n'utilise que deux boucles impbriquées itérant sur l'entièreté des séquences. Dès lors, la complexité en temps est
		\begin{equation}
			\Theta\rbk{n^2}
		\end{equation}
		Concernant la complexité en espace, la matrice de coût possède $(n + 1)^2$ éléments. La complexité spatiale est donc aussi $\Theta\rbk{n^2}$.
	\end{enumerate}
	\newpage
\end{document}
